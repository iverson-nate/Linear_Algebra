\section{Geometry of $\mathbb{R}^n$}

Each vector $\begin{bmatrix}v_1\\v_2 \\ \vdots \\ v_n \end{bmatrix} \in \mathbb{R}^n$ can be visualized as an arrow going from the origin 
$(0,0, \ldots, 0)$ to the point $(v_1, v_2, \ldots, v_n)$.  In this case we'd call the \textbf{base point} the origin.

\begin{example}
$\begin{bmatrix}2\\5\end{bmatrix}$ can be visualized as the arrow between $(0,0)$ and $(2,5)$
\begin{figure}[ht]
  \centering
  \begin{tikzpicture}
    \coordinate (Origin)   at (0,0);
    \coordinate (XAxisMin) at (-2,0);
    \coordinate (XAxisMax) at (6,0);
    \coordinate (YAxisMin) at (0,-1);
    \coordinate (YAxisMax) at (0,6);
    \draw [thin, gray,-latex] (XAxisMin) -- (XAxisMax);% Draw x axis
    \draw [thin, gray,-latex] (YAxisMin) -- (YAxisMax);% Draw y axis
    \draw [ultra thick,-latex,red] (Origin) -- (2,5) node [below right] {$(2,5)$};
  \end{tikzpicture}
\end{figure}
\end{example}

We can also think of any arrow from point $(a_1, a_2, \ldots, a_n)$ to point $(b_1, b_2, \ldots, b_n)$ as a vector. We've simply changed 
the \text{base point} to $(a_1, a_2, \ldots, a_n)$ instead of the origin. 

\begin{definition}
The \textbf{canonical form} of a vector from $(a_1, a_2, \ldots, a_n)$ to $(b_1, b_2, \ldots, b_n)$ is the vector 
$\begin{bmatrix}b_1-a_1 \\ b_2-a_2 \\ \vdots \\ b_n-a_n\end{bmatrix}$
\end{definition}

\begin{remark}
You can think of the canonical form of a vector is taking an arrow with the same length and direction with base point of the origin.
\end{remark}

\begin{example}

\end{example}
