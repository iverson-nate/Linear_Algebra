\subsubsection{Exercises}
\begin{exercise}For matrix 
$A=\begin{bmatrix*}[C]
1  & 2  \\
0  & -1  \\
-1 & 0  \\
\end{bmatrix*}$
use the definition, as in \\Examples~\ref{ex:mv_mult_by_def}
and \ref{ex:mv_mult_by_def_gen}, to  multiply by the 
following vectors (if not possible explain why):\\
\begin{inparaenum}[a.)]
\item $\begin{bmatrix*}[C]1 \\ -2 \\ -1 \end{bmatrix*}$\quad 
\item $\begin{bmatrix*}[C]2 \\ -3 \end{bmatrix*}$\quad 
\item $\begin{bmatrix*}[C]0 \\ 0 \\ 0 \end{bmatrix*}$\quad
\item $\begin{bmatrix*}[C]v_1 \\ v_2 \end{bmatrix*}$
\end{inparaenum}
\end{exercise}

\begin{exercise}For matrix 
$A=\begin{bmatrix*}[C]
1  & 2  & 3 & 4 & 5 \\
0  & -1 & 2 & 0 & 3 \\
-1 & 0  & 0 & -1& 1 \\
\end{bmatrix*}$
use the \textbf{Row-Column Algorithm}, as in 
Proposition~\ref{ex:mv_mult_by_def}
to  multiply by the following vectors (if not possible explain why):\\
\begin{inparaenum}[a.)]
\item $\begin{bmatrix*}[C]1 \\ -2 \\ -1 \\ 2 \\ 3\end{bmatrix*}$\quad 
\item $\begin{bmatrix*}[C]1 \\ -2 \\ -1 \\ 2\end{bmatrix*}$\quad 
\item $\begin{bmatrix*}[C]0 \\ 0 \\ 3 \\ 0 \\ 1\end{bmatrix*}$\quad
\item $\begin{bmatrix*}[C]v_1 \\ v_2 \\ v_3 \\ v_4 \\ v_5\end{bmatrix*}$
\end{inparaenum}
\end{exercise}

\begin{exercise} Consider
$A=\begin{bmatrix*}[C]
1 & 3  \\
0 & -1 \\
-3 & 2 \\ 
\end{bmatrix*}$.\\
\begin{inparaenum}[a.)]
\item If $A\vec{v}$ is possible, for what value of $n$ is 
$\vec{v} \in \mathbb{R}^n$? (Why?)\\
%
\item If $A\vec{v}$ is possible, for what value of $m$ is 
$A\vec{v} \in \mathbb{R}^m$? (Why?)\\
%
\item Pick a generic vector $\vec{v} \in \mathbb{R}^n$ and write it as 
$\vec{v}=\begin{bmatrix}v_1 \\ v_2 \\ \vdots \\ v_n\end{bmatrix}$ where 
$n$ is from part a.). What is the value of $A\vec{v}$?
\end{inparaenum}
\end{exercise}

\begin{exercise}
Let $A=\begin{bmatrix*}[C]
1 & 1  \\
0 & -3  \\
-1 & 1  
\end{bmatrix*}$.\\
\begin{inparaenum}
\item If possible, find a vector $\vec{v}$ such that 
$A\vec{v}=\begin{bmatrix*}[C] 3 \\ -6 \\ 2\end{bmatrix*}$.\\
%
\item If possible, find a vector $\vec{v}$ such that 
$A\vec{v}=\begin{bmatrix*}[C] 2 \\ -3 \\ 1\end{bmatrix*}$.
\end{inparaenum}
\end{exercise}

\begin{exercise}
Let $A=\begin{bmatrix*}[C]
2 & 1  & -3\\
0 & -3 & 1 \\
0 & 0 & 1 
\end{bmatrix*}$.\\
\begin{inparaenum}
\item If possible, find a vector $\vec{v}$ such that 
$A\vec{v}=\begin{bmatrix}1 \\ 2 \\ 3\end{bmatrix}$.\\
%
\item Given any $\vec{b} \in \mathbb{R}^3$ can you always find a 
$\vec{v} \in \mathbb{R}^3$ such that $A\vec{v}=\vec{b}$? (Why?)
\end{inparaenum}
\end{exercise}
