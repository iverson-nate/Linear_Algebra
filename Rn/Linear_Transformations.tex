\section{Linear Transformations in $\mathbb{R}^n$}
\index{Linear Transformations}

\begin{definition}[Function]
A function $f:A \to B$ is a map the associates each value $a$ in the  
\emph{domain} $A$ with exactly one value $b$ in the \emph{codomain} $B$. 
We would write $f(a)=b$. We will call all the elements in $B$ actually mapped 
to by $f$ the \emph{image} of $A$ under $f$ and denote it 
$f(A)=\{ b \in B \mid f(a)=b\text{ for some } a \in A  \}$.
\end{definition}
\begin{example}$f:\mathbb{R} \to \mathbb{R}$ defined by $f(x)=x^2$ is a function. The domain  and range are both $\mathbb{R}$ where the image $f(\mathbb{R})=[0,\infty )$.
\end{example}
\begin{remark}
We are intentionally avoiding the term \emph{range} of a function because it is ambiguous. Some authors call the range the image and others call it the codomain.
\end{remark}
\begin{definition}[Linear Transformation]
A real \emph{linear transformation} is a function\\ $T: \mathbb{R}^n \to \mathbb{R}^m$ such that vector addition and scalar multiplication are preserved.
That is 
\begin{itemize}
\item $T(\vec{v}+\vec{w})=T(\vec{v})+T(\vec{w})$ for all $\vec{v},\vec{w} \in \mathbb{R}^n$\hfill \emph{(preserves vector addition)}
\item $T(a\vec{v})=aT(\vec{v})$ for all $\vec{v}\in \mathbb{R}^n$ and for $a \in \mathbb{R}$\hfill \emph{(preserves scalar multiplication)}
\end{itemize}
\end{definition}

\begin{proposition} If $A \in M_{m \times n}(\mathbb{R})$ then the function $T:\mathbb{R}^n \to \mathbb{R}^m$ defined by $T(\vec{v})=A\vec{v}$ is a linear transformation. This map is commonly refered to as $\vec{v} \to A\vec{v}$.
\end{proposition}
\begin{proof}
Let $\vec{v},\vec{w} \in \mathbb{R}^n$ and $a \in \mathbb{R}$. 

We will start with scalar multiplication 

\begin{align*}
T(a\vec{v})
%=A\begin{bmatrix}av_1\\ \vdots \\ av_n\end{bmatrix}
&=[\vec{a}_1, \ldots, \vec{a}_n]
\begin{bmatrix}av_1\\ \vdots \\ av_n\end{bmatrix}\\
&=(av_1)\vec{a}_1+\cdots+(av_n)\vec{a}_n\\
&=a(v_1\vec{a}_1+\cdots+v_n\vec{a}_n)\\
&=a(A\vec{v})\\
&=aT(\vec{v})\\
\end{align*}


Now we show vector addition is preserved:


\begin{align*}
T(\vec{v}+\vec{w})
%&=A\begin{bmatrix}v_1+w_1\\ \vdots \\ v_n+w_n\end{bmatrix}\\
&=[\vec{a}_1, \ldots, \vec{a}_n]\begin{bmatrix}v_1+w_1\\ \vdots \\ v_n+w_n\end{bmatrix}\\
&=(v_1+w_1)\vec{a}_1+\cdots+(v_n+w_n)\vec{a}_n\\
&=(v_1\vec{a}_1+\cdots+v_n\vec{a}_n)+(w_1\vec{a}_1+\cdots+w_n\vec{a}_n)\\
&=A\vec{v}+A\vec{w}\\
&=T(\vec{v})+T(\vec{w})
\end{align*}

Note that the vector associative and distributive properties used above 
follow from the associative and distributive properties in $\mathbb{R}$ 
by applying them on each coordinate.

Since the choices of $\vec{v},\vec{w}$ and $a$ were arbitrary the proof works for all $\vec{v},\vec{w} \in \mathbb{R}^n$, $a \in \mathbb{R}$.

\end{proof}
