\subsubsection{Exercises}
%\addcontentsline{toc}{subsubsection}{Exercises}
\begin{exercise}
$A=\begin{bmatrix*}[C]
1  & 0  & 3  & -5\\
3  & 2  & 0  & 7 \\
-1 & -2 & -1 & 0
\end{bmatrix*}$\\
\begin{inparaenum}[a.)]
\item How many rows does $A$ have?\\
\item How many columns does $A$ have?\\
\item What are the dimensions of $A$. That is $m \times n$ such that 
$A \in M_{m\times n}(\mathbb{R})$.\\
\item Find the column vectors $\{\vec{a}_1,\ldots,\vec{a}_n\}$. That is
$A=[\vec{a}_1,\ldots,\vec{a}_n]$.\\
\item $A=[a_{ij}]$ Find the following values $a_{13}$, $a_{31}$, $a_{24}$. 
\end{inparaenum}
\end{exercise}
\begin{exercise}
Let $A,B,C \in M_{m\times n}$ and $r,t \in \mathbb{R}$. Show 
the following:\\
\begin{inparaenum}[a.)]
\item Scalar multiplication identity. That is, $1 A=A$\\
\item Matrix addition is commutative. That is, 
$A+B=B+A$.\\
\item Matrix addition is associative. That is, 
$(A+B)+C=A+(B+C)$\\
\item The scalar product is compatible. That is, $(rt)A=r(tA)$\\
\item Additive identity. That is there is a $0\in M_{m\times n}$ such that
 $0+A=A+0=A$\\
\item Additive inverse. That is, there is a $-A \in M_{m\times n}$ such 
that $A+(-A)=(-A)+A=0$.
Also show that $(-A)=(-1)A$. \\
\item Distributive properties of scalar product. That is, \\
\begin{inparaenum}[i.)]
\indent \item  $r(A+B)=rA+rB$ and,\\
\indent \item  $(r+t)A=rA+tA$.\\
\end{inparaenum}
\end{inparaenum} 
\end{exercise}
