\subsubsection{Exercises}
%\addcontentsline{toc}{subsubsection}{Exercises}

\begin{exercise}
Compute the following:\\
\begin{inparaenum}[a.)]
\begin{tabular}{llll}
\item $3 \begin{bmatrix*}[C] 1 \\ 0\\ -5\end{bmatrix*}$ \hspace{1cm} &
\item $\begin{bmatrix*}[C]1 \\ 0\\ -5\end{bmatrix*} + \begin{bmatrix}4 \\ 1\\ 7\end{bmatrix}$ \hspace{1cm} & 
\item $3\begin{bmatrix*}[C]1 \\ 0\\ -5\end{bmatrix*} + 2\begin{bmatrix}4 \\ 1\\ 7\end{bmatrix}$ \hspace{1cm} &
\item $a\begin{bmatrix*}[C]1 \\ 0\\ -5\end{bmatrix*} + b\begin{bmatrix}4 \\ 1\\ 7\end{bmatrix}$
\end{tabular}
\end{inparaenum}
\end{exercise}

\begin{exercise}
For $0 \in \mathbb{R}$ and $\vec{v} \in \mathbb{R}^n$ show that $0\vec{v}=\vec{0} \in \mathbb{R}^n$.
\end{exercise}


\begin{exercise}
Let $\vec{v},\vec{w},\vec{x} \in \mathbb{R}^n$ and $r,t \in \mathbb{R}$. Show 
the following:
\begin{enumerate}[label=\alph*.)]
 \item Scalar multiplication identity. That is, $1 \vec{v}=\vec{v}$
\item Vector addition is commutative. That is, 
$\vec{v}+\vec{w}=\vec{w}+\vec{v}$.
\item Vector addition is associative. That is, 
$(\vec{v}+\vec{w})+\vec{x}=\vec{w}+(\vec{v}+\vec{x})$.
\item The scalar product is compatible. That is, $(rt) \vec{v}=r(t \vec{v})$
\item Additive identity. That is, $\vec{0}+\vec{v}=\vec{v}+\vec{0}=\vec{v}$
\item Additive inverse. That is, there is a $-\vec{v} \in \mathbb{R}^n$ such 
that $\vec{v}+(-\vec{v})=(-\vec{v}+\vec{v})=\vec{0}$. 
Also show that $(-\vec{v})=(-1)\vec{v}$. 
\item Distributive properties of scalar product. That is, 
\begin{enumerate}[label=\roman*.)]
 \item  $r(\vec{v}+\vec{w})=r\vec{v}+r\vec{w}$ and,
\item  $(r+t)\vec{v}=r\vec{v}+t\vec{v}$.
\end{enumerate}
\end{enumerate} 
\end{exercise}


\begin{exercise}
Show that every vector in $\mathbb{R}^3$ can be written as a linear combination 
of the following vectors: $\begin{bmatrix}1\\ 0 \\ 0\end{bmatrix}, 
\begin{bmatrix}1\\ 1 \\ 0\end{bmatrix},\begin{bmatrix}1\\ 1 \\ 1\end{bmatrix}$
\end{exercise}


