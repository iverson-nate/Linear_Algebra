\subsubsection{Exercises}

\begin{exercise} Find the canonical form of each of the following vectors:\\
\begin{inparaenum}[a)]
\item a vector from $(0,0)$ to $(1,3)$ in $\mathbb{R}^2$\\
\item a vector from $(1,2)$ to $(5,6)$ in $\mathbb{R}^2$\\
\item a vector from $(0,0,1)$ to $(-3,1,5)$ in $\mathbb{R}^3$\\
\item a vector from $(3,0,-2)$ to $(0,0,0)$ in $\mathbb{R}^3$\\
\end{inparaenum}
\end{exercise}

\begin{exercise} Draw a lattice of enpoints of vectors $\vec{u}=\begin{bmatrix}2\\1\end{bmatrix}$ and $\vec{v}=\begin{bmatrix}1\\-5\end{bmatrix}$. That is graph 
$m\vec{u}+n\vec{v}$ for integer values of $m$ and $n$. Use the lattice to graphically determine, or estimate, the coefficients $r$ and $t$ for the linear combination of $r\vec{u}+t\vec{v}$ that gives the vector below:\\
\begin{inparaenum}
\item $\begin{bmatrix}3 \\ 7 \end{bmatrix}$\hfill 
\item $\begin{bmatrix*}[C]-1 \\ -6 \end{bmatrix*}$\hfill {} \\
\item $\begin{bmatrix*}[C]-9 \\ 1 \end{bmatrix*}$\hfill 
\item $\begin{bmatrix*}[C] 0 \\ -5.5 \end{bmatrix*}$\hfill {} \\
\end{inparaenum}
\end{exercise}

\begin{exercise} For each of the linear transformations below, write its image as a span of vectors. Find the standard matrix if necessary.\\
\begin{inparaenum}[a)]
\item $T\begin{bmatrix}a\\b\\c\end{bmatrix}=
\begin{bmatrix*}[C]
1 & 2 & 0\\
-1 & 0 & -3
\end{bmatrix*}
\begin{bmatrix}a \\ b \\ c \end{bmatrix}$ \hfill
\item $S(a)=\begin{bmatrix*}[C] 7a \\ 0 \\ -a \end{bmatrix*}$ \hfill {} \\
\item $S\begin{bmatrix}a\\b\\c\end{bmatrix}=\begin{bmatrix}3a-2b \\ 2a-c \\ a \\ b-c \\ c \end{bmatrix}$ \hfill
\item $T\begin{bmatrix} y_1 \\ y_2 \end{bmatrix} = \begin{bmatrix} y_1 \\ 3 y_1 \end{bmatrix}$ \hfill {} \\
\item $R\begin{bmatrix}x \\ y \\ z \end{bmatrix} = 3x+2y-z$ \hfill 
\item $M\begin{bmatrix}x \\ y \\ z \end{bmatrix} = 
\begin{bmatrix*}[C]
0 &1 & -3 \\
0 &-1 & 3 \\
\end{bmatrix*}\begin{bmatrix}x \\ y \\ z \end{bmatrix} 
$ \hfill {} \\
\item $R\begin{bmatrix}a \\ b \\ c \end{bmatrix} =\begin{bmatrix*}[C] 2a \\ 3b \\ 0 \\ -c \end{bmatrix*}$ \hfill
\item $M \begin{bmatrix}x_1\\ x_2 \\ x_3\end{bmatrix}=\begin{bmatrix}x_2\\ x_1 \\ x_3\end{bmatrix}$ \hfill {} \\
\end{inparaenum}
\end{exercise}
