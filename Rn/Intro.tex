\section{Introduction}
\index{$\mathbb{R}^n$, Introduction}
\subsection{$\mathbb{R}^n$}
We will assume that all readers are already familiar with a concept of real and complex numbers. 
In this text we denote the real numbers by $\mathbb{R}$ and the complex numbers by $\mathbb{C}$. 

\begin{definition}
We define $\mathbb{R}^n$ to be the set of real column vectors in $n$-coordinates. That is 
\[
\mathbb{R}^n=\left\{ 
\begin{bmatrix}
x_1\\ x_2 \\ \vdots \\ x_n
\end{bmatrix} 
: x_i \in \mathbb{R}\text{ for all }i=1,2, \ldots, n \right\}
\]
We will refer to the members of $\mathbb{R}^n$ as \emph{vectors} and the 
elements of $\mathbb{R}$ as \emph{scalars}. When we want to refer to row 
vectors instead of column vectors we will use the notation 
$\mathbb{R}^n_\text{row}$. The vector consisting of all zero coordinates is 
called the \emph{zero vector} and denoted $\vec{0}$.
\end{definition}

For convenience we may refer to each $\vec{x} \in \mathbb{R}^n$ as a single 
bold letter where we will for allow each coordinate be labeled as the same 
subscriped letter. For example if $\vec{x}, \vec{y}, \vec{z}, \vec{0} \in \mathbb{R}^n$
\[
\vec{x}=\begin{bmatrix}x_1\\ x_2 \\ \vdots \\ x_n\end{bmatrix}
\vec{y}=\begin{bmatrix}y_1\\ y_2 \\ \vdots \\ y_n\end{bmatrix}
\vec{v}=\begin{bmatrix}v_1\\ v_2 \\ \vdots \\ v_n\end{bmatrix}
\vec{0}=\begin{bmatrix}0\\ 0 \\ \vdots \\ 0\end{bmatrix}
\]

\begin{remark}
Sometimes it is useful to describe a vector as a function on it's coordinates. 
For example if 
$\vec{v}=\begin{bmatrix}1 \\ 4 \\ 9 \end{bmatrix}$ We may say 
$\vec{v}(2)=4$ that is the \nth{2} coordinate of $\vec{v}$ is $4$.
\end{remark}


\begin{definition}[Vector Equality]\index{vector equality}
Vectors $\vec{v}, \vec{w} \in \mathbb{R}^n$ are \emph{equal},  if they are equal 
coordinatewise. That is, $v_i=v(i)=w(i)=w_i$ for all $i=1,2,\ldots, n$. We 
denote this with  $\vec{v}=\vec{w}$.
\end{definition}


\begin{definition}[Vector Addition]\index{vector addition}
We define \emph{vector addition} on $\mathbb{R}^n$ coordinatewise. 
That is, for $\vec{v},\vec{w} \in \mathbb{R}^n$ we define
\[
\vec{v}+\vec{w}=
\begin{bmatrix}v_1\\ v_2 \\ \vdots \\ v_n\end{bmatrix}+
\begin{bmatrix}w_1\\ w_2 \\ \vdots \\ w_n\end{bmatrix}=
\begin{bmatrix}v_1+w_1\\ v_2+w_2 \\ \vdots \\ v_n+w_n\end{bmatrix}
\]
\end{definition}

\begin{definition}[Scalar Multiplication]\index{scalar multiplication}
We define \emph{scalar multiplication} on $\mathbb{R}^n$ coordinatewise. That 
is, for $\vec{v} \in \mathbb{R}^n$ and $r \in \mathbb{R}$ we define
\[
r\vec{v}=
r\begin{bmatrix}v_1\\ v_2 \\ \vdots \\ v_n\end{bmatrix}=
\begin{bmatrix}rv_1\\ rv_2 \\ \vdots \\ rv_n\end{bmatrix}
\]
\end{definition}

The order of operations is scalar multiplication first then vector addition.

\begin{example}
Consider $\vec{x}=\begin{bmatrix*}[C]1\\ 0 \\ -3\end{bmatrix*}$ and
$\vec{y}=\begin{bmatrix*}[C]-4\\ 2 \\ 1\end{bmatrix*}$ in $\mathbb{R}^3$.
\[
2\vec{x}+(-1)\vec{y}=
2\begin{bmatrix*}[C]1\\ 0 \\ -3\end{bmatrix*}+
(-1)\begin{bmatrix*}[C]-4\\ 2 \\ 1\end{bmatrix*}=
%\begin{bmatrix}2(1)\\ 2(0) \\ 2(-3)\end{bmatrix}+
%\begin{bmatrix}(-1)(-4)\\ (-1)(2) \\ (-1)(1)\end{bmatrix}=
\begin{bmatrix*}[C]2\\ 0 \\ -6\end{bmatrix*}+
\begin{bmatrix*}[C]4\\ -2 \\ -1\end{bmatrix*}=
\begin{bmatrix}2+4\\ 0+(-2) \\ -6+(-1)\end{bmatrix}=
\begin{bmatrix*}[C]6\\ -2 \\ -5\end{bmatrix*}
\]
This is actually also an example of our next definition.
\end{example}

\begin{definition}\index{linear combination}
A \emph{linear combination} of vectors $\vec{a}_1, \vec{a}_2, \ldots,
\vec{a}_n$ is the sum
\[c_1\vec{a}_1+c_2\vec{a}+\cdots+c_n\vec{a}_n\]
for some choice of scalar multiples $c_1,c_2, \ldots, c_n \in \mathbb{R}$
\end{definition}


\begin{example}
Any vector in $\mathbb{R}^2$ can be written as a linear combination of
$\begin{bmatrix}1 \\ 0\end{bmatrix}$ and $\begin{bmatrix}1 \\ 1 \end{bmatrix}$
because
\[
\begin{bmatrix}v_1 \\ v_2\end{bmatrix}=
(v_1-v_2) \begin{bmatrix}1 \\ 0\end{bmatrix}+
v_2 \begin{bmatrix}1 \\ 1 \end{bmatrix}
\]
\end{example}

\begin{example}
The vector $\begin{bmatrix}1 \\ 1\end{bmatrix}$ is not a linear combination
of $\begin{bmatrix}1 \\ 0\end{bmatrix}$ and $\begin{bmatrix}2 \\
0\end{bmatrix}$. If it were then, we would have
\[
\begin{bmatrix}1 \\ 1\end{bmatrix}=
a \begin{bmatrix}1 \\ 0\end{bmatrix}+
b \begin{bmatrix}2 \\ 0 \end{bmatrix}
\]
for some constants $a,b\in \mathbb{R}$. Considering the second components, this
implies that $1=0$ which is a contradiction.
\end{example}

\begin{definition}
\index{standard basis for $\mathbb{R}^n$}
The \emph{standard basis for} $\mathbb{R}^n$ is a set of $n$-vectors
$\{\vec{e}_1, \vec{e}_2, \ldots, \vec{e}_n\}$ such that\\
$\vec{e}_j(k)=\begin{cases}
1 & \text{ if } k=j\\
0 & \text{ if } k\neq j
\end{cases}$
for each $j,k$ with $1\le j \le n$ and $1 \le k \le n$.
\end{definition}


\begin{remark}For each integer $n\geq 1$ the standard basis of $\mathbb{R}^n$ 
can also be written as $n$ vectors:
\[
\vec{e}_{1}=\begin{bmatrix} 1 \\  0 \\ 0 \\ \vdots \\ \vdots \\ 0 \\ 0\end{bmatrix},
\vec{e}_{2}=\begin{bmatrix} 0 \\  1 \\ 0 \\ \vdots \\ \vdots \\ 0 \\ 0\end{bmatrix},
\vec{e}_{3}=\begin{bmatrix} 0 \\  0 \\ 1 \\ \vdots \\ \vdots \\ 0 \\ 0\end{bmatrix},
\cdots,
\vec{e}_k=\begin{bmatrix} 0 \\  \vdots \\ 0 \\ 1 \\ 0 \\ \vdots \\ 0 \end{bmatrix}
\begin{array}{l}
\phantom{0} \\  \phantom{\vdots} \\ \phantom{0} \\ \gets k^{\tiny{\text{th}}} 
\text{ coordinate, } \\ \phantom{0} \\ \phantom{\vdots} \\ \phantom{0} 
\end{array} 
\cdots,
\vec{e}_{n-1}=\begin{bmatrix} 0 \\  0 \\ 0 \\ \vdots \\ \vdots \\ 1 \\ 0\end{bmatrix},
\vec{e}_{n}=\begin{bmatrix} 0 \\  0 \\ 0 \\ \vdots\\ \vdots \\ 0 \\ 1\end{bmatrix}
\]
\end{remark}
\begin{example}
In $\mathbb{R}^3 $ every vector is a linear combination of the standard basis 
$\{\vec{e}_1,\vec{e}_2,\vec{e}_3\}$
\[
\begin{bmatrix}v_1 \\ v_2 \\ v_3\end{bmatrix}=
v_1 \begin{bmatrix}1 \\ 0 \\ 0\end{bmatrix}+
v_2 \begin{bmatrix}0 \\ 1 \\ 0\end{bmatrix}+
v_3 \begin{bmatrix}0 \\ 0 \\ 1\end{bmatrix}
=v_1\vec{e}_1+v_3\vec{e}_3+v_3\vec{e}_3
\]
\end{example}

\begin{proposition}\label{prop:e_k_spans_Rn}
In $\mathbb{R}^n$ every vector is a linear combination of the standard basis 
$\{\vec{e}_1, \vec{e}_2, \ldots, \vec{e}_n\}$.
\end{proposition}

\begin{proof}
Let $\vec{v} \in \mathbb{R}^n$ and $1\leq k\leq n$. Then we observe that  
\[\vec{v}(k)=v_k=v_k\cdot 1=v_k \vec{e}_k(k).\]
We can rewrite this equation by adding $k-1$ zeros before and $n-k$ zeros 
after $\vec{v}(k)$. That is,  
\[\vec{v}(k)=0+ \cdots + 0+v_k\vec{e}_k(k)+0+\cdots+0.\] 
Since $\vec{e}_j(k)=0$ for all $j\neq k$ we have 
\[\vec{v}(k)=v_1\vec{e}_1(k)+v_2\vec{e}_2(k)+\cdots+ v_k\vec{e}_k(k)+ \cdots + 
v_n\vec{e}_n(k).\]
Since the above equation is true for any $1\leq k\leq n$ we have,  
\[\vec{v}=v_1\vec{e}_1+v_2\vec{e}_2+\cdots+ v_n\vec{e}_n.\]
\end{proof}

\subsubsection{Exercises}
\addcontentsline{toc}{subsection}{Exercises}

\begin{exercise}
Compute the following:\\
\begin{inparaenum}[a.)]
\begin{tabular}{llll}
\item $3 \begin{bmatrix*}[C] 1 \\ 0\\ -5\end{bmatrix*}$ \hspace{1cm} &
\item $\begin{bmatrix*}[C]1 \\ 0\\ -5\end{bmatrix*} + \begin{bmatrix}4 \\ 1\\ 7\end{bmatrix}$ \hspace{1cm} & 
\item $3\begin{bmatrix*}[C]1 \\ 0\\ -5\end{bmatrix*} + 2\begin{bmatrix}4 \\ 1\\ 7\end{bmatrix}$ \hspace{1cm} &
\item $a\begin{bmatrix*}[C]1 \\ 0\\ -5\end{bmatrix*} + b\begin{bmatrix}4 \\ 1\\ 7\end{bmatrix}$
\end{tabular}
\end{inparaenum}
\end{exercise}

\begin{exercise}
For $0 \in \mathbb{R}$ and $\vec{v} \in \mathbb{R}^n$ show that $0\vec{v}=\vec{0} \in \mathbb{R}^n$.
\end{exercise}


\begin{exercise}
Let $\vec{v},\vec{w},\vec{x} \in \mathbb{R}^n$ and $r,t \in \mathbb{R}$. Show the following:\\
\begin{inparaenum}[a.)]
\item Scalar multiplication identity. That is $1 \vec{v}=\vec{v}$\\
\item Vector addition is commutative. That is $\vec{v}+\vec{w}=\vec{w}+\vec{v}$.\\
\item Vector addition is associative. That is $(\vec{v}+\vec{w})+\vec{x}=\vec{w}+(\vec{v}+\vec{x})$.\\
\item The sclar product is compatible. That is $(rt) \vec{v}=r(t \vec{v})$\\
\item Additive identity. That is $\vec{0}+\vec{v}=\vec{v}+\vec{0}=\vec{v}$\\
\item Additive inverse. That is there is a $-\vec{v} \in \mathbb{R}^n$ such that $\vec{v}+(-\vec{v})=(-\vec{v}+\vec{v})=\vec{0}$. \\
Also show that $(-\vec{v})=(-1)\vec{v}$.
\item Distributive properties of scalar product. That is\\
\begin{inparaenum}[\ \ \ \ i.)]
\item  $r(\vec{v}+\vec{w})=r\vec{v}+r\vec{w}$ and,\\
\item  $(r+t)\vec{v}=r\vec{v}+t\vec{v}$.\\
\end{inparaenum}
\end{inparaenum} 
\end{exercise}


\begin{exercise}
Show that every vecotor in $\mathbb{R}^3$ can be written as a linear combination of the following vectors: $\begin{bmatrix}1\\ 0 \\ 0\end{bmatrix}, \begin{bmatrix}1\\ 1 \\ 0\end{bmatrix},\begin{bmatrix}1\\ 1 \\ 1\end{bmatrix}$
\end{exercise}




\subsection{Real Matrices}
\index{Real Matrices}

\begin{definition}[Matrix]
A real matrix $A$ is a rectangular array of real numbers 
$a_{i,j} \in \mathbb{R}$ (usually the comma is omitted $a_{ij}$)  where the $i$ 
is the row and $j$ is the column number. That is, 
\[
A=[a_{ij}]=\begin{bmatrix}
a_{11} & a_{12} & \cdots & a_{1n} \\
a_{21} & a_{22} & \cdots & a_{2n} \\
\vdots & \vdots & \ddots & \vdots \\
a_{m1} & a_{m2} & \cdots & a_{mn} 
\end{bmatrix}
\]
The \emph{dimension} of the matrix is $m\times n$ where $m$ is the number of 
rows and $n$ is the number of columns. The set of all such matrices is denoted 
by $M_{m\times n}(\mathbb{R})$. 
\end{definition}
The matrix $A$ can be also seen as a collection of vectors. Namely, 
$A=[a_{ij}]=[\vec{a}_1, \vec{a}_2, \ldots, \vec{a}_n]$ where 
 $\vec{a}_1, \vec{a}_2,\ldots, \vec{a}_n \in \mathbb{R}^n$ are the \emph{column 
vectors} of $A$. In particular, 
\[
\vec{a}_1=\begin{bmatrix}a_{11}\\a_{21}\\ \vdots \\ a_{m1}\end{bmatrix}, \quad 
\vec{a}_2=\begin{bmatrix}a_{12}\\a_{22}\\ \vdots \\ a_{m2}\end{bmatrix}, 
\cdots, \quad 
\vec{a}_k=\begin{bmatrix}a_{1k}\\a_{2k}\\ \vdots \\ a_{mk}\end{bmatrix}, 
\cdots, \quad 
\vec{a}_n=\begin{bmatrix}a_{1n}\\a_{2n}\\ \vdots \\ a_{mn}\end{bmatrix}
\]

\begin{example}
$A=\begin{bmatrix*}[C]
2 & -3 & 0 \\
-5 & 0 & 1 
\end{bmatrix*}$
is a $2 \times 3$ matrix, $a_{12}=-3$, and  
$\vec{a}_1=\begin{bmatrix*}[C] 2 \\ -5 \end{bmatrix*} $
\end{example}


\subsubsection{Exercises}
%\addcontentsline{toc}{subsubsection}{Exercises}
\begin{exercise}
$A=\begin{bmatrix*}[C]
1  & 0  & 3  & -5\\
3  & 2  & 0  & 7 \\
-1 & -2 & -1 & 0
\end{bmatrix*}$\\
\begin{inparaenum}[a.)]
\item How many rows does $A$ have?\\
\item How many columns does $A$ have?\\
\item What are the dimensions of $A$. That is $m \times n$ such that 
$A \in M_{m\times n}(\mathbb{R})$.\\
\item Find the column vectors $\{\vec{a}_1,\ldots,\vec{a}_n\}$. That is
$A=[\vec{a}_1,\ldots,\vec{a}_n]$.\\
\item $A=[a_{ij}]$ Find the following values $a_{13}$, $a_{31}$, $a_{24}$. 
\end{inparaenum}
\end{exercise}
\begin{exercise}
Let $A,B,C \in M_{m\times n}$ and $r,t \in \mathbb{R}$. Show 
the following:\\
\begin{inparaenum}[a.)]
\item Scalar multiplication identity. That is, $1 A=A$\\
\item Matrix addition is commutative. That is, 
$A+B=B+A$.\\
\item Matrix addition is associative. That is, 
$(A+B)+C=A+(B+C)$\\
\item The scalar product is compatible. That is, $(rt)A=r(tA)$\\
\item Additive identity. That is there is a $0\in M_{m\times n}$ such that
 $0+A=A+0=A$\\
\item Additive inverse. That is, there is a $-A \in M_{m\times n}$ such 
that $A+(-A)=(-A)+A=0$.
Also show that $(-A)=(-1)A$. \\
\item Distributive properties of scalar product. That is, \\
\begin{inparaenum}[i.)]
\indent \item  $r(A+B)=rA+rB$ and,\\
\indent \item  $(r+t)A=rA+tA$.\\
\end{inparaenum}
\end{inparaenum} 
\end{exercise}


\subsection{Matrix-Vector Multiplication}

\begin{definition}[Matrix-Vector Multiplication]\index{Matrix-Vector Multiplication}
Let $A=[a_{ij}]\in M_{m\times n}(\mathbb{R})$ and $\vec{v} \in \mathbb{R}^n$.  
We define the \emph{Matrix-Vector} product to be the linear combination 
\[
A\vec{v}=[\vec{a}_1, \vec{a}_2, \ldots ,\vec{a}_n]\begin{bmatrix}v_1 \\ v_2 \\ 
\vdots \\ v_n \end{bmatrix}=v_1\vec{a}_1+v_2\vec{a}_2+\cdots+v_n\vec{a}_n
\]
\end{definition} 
\begin{remark}
It is important to note that the product is only defined if the number of columns in $A$ matches the number of coordinates in $\vec{v}$.
\end{remark}

\begin{example}
\begin{align*}
\begin{bmatrix*}[C]
2 & -3 & 0 \\
-5 & 0 & 1 
\end{bmatrix*} 
\begin{bmatrix}1\\ 2 \\ 3 \end{bmatrix}
&=1\begin{bmatrix*}[C]2\\-5\end{bmatrix*} 
+2\begin{bmatrix*}[C]-3\\0\end{bmatrix*} 
+3\begin{bmatrix}0\\1\end{bmatrix}\\
&=\begin{bmatrix*}[C]2\\-5\end{bmatrix*}
+\begin{bmatrix*}[C]-6\\0\end{bmatrix*}+\begin{bmatrix}0\\3\end{bmatrix}\\
&=\begin{bmatrix*}[C]-4\\-2\end{bmatrix*}
\end{align*}
\end{example}

\begin{example}
\begin{align*}
\begin{bmatrix*}[C]
2 & -3 & 0 \\
-5 & 0 & 1 
\end{bmatrix*}
\begin{bmatrix}v_1\\ v_2 \\ v_3 \end{bmatrix}
&=v_1\begin{bmatrix*}[C]2\\-5 \end{bmatrix*}
+v_2\begin{bmatrix*}[C]-3 \\0\end{bmatrix*}
+v_3\begin{bmatrix}0\\1\end{bmatrix}\\
&=\begin{bmatrix*}[C]2v_1\\-5v_1\end{bmatrix*}
+\begin{bmatrix}-3v_2\\ 0\end{bmatrix} 
+\begin{bmatrix}0\\v_3\end{bmatrix}\\
&=\begin{bmatrix}2v_1-3v_2\\-5v_1+v_3\end{bmatrix}\\
&=\begin{bmatrix}2v_1-3v_2+0v_3\\-5v_1+0v_2+v_3\end{bmatrix}
\end{align*}
Notice that we put some zero values back in on the final result. This is not 
strictly necessary but helps motivate the row-column algorithm for matrix 
multiplication.
\end{example}

\begin{proposition}[The Row-Column Algorithm] 
Let $A=[a_{ij}]\in M_{m\times n}(\mathbb{R})$ and $\vec{v} \in \mathbb{R}^n$. 
Then $\vec{w}=A\vec{v} \in \mathbb{R}^m$ where  
\[\vec{w}(k)=a_{k1}v_1+a_{k2}v_2+\cdots+a_{kn}v_n\] 
for all $k$ with $1 \le k \le n$.
\end{proposition}
\begin{remark} 
This can be visualized by taking each row in $A$ and multiplying 
each of it's entries by the corresponding coordinate in $\vec{v}$. The sum of 
these entries is the same row in the result as the row in $A$.\\
% tikz picture giving vector row-product illustration.
%
\usetikzlibrary{matrix,shapes,arrows,decorations.pathmorphing}
\begin{tikzpicture}
% les matrices
\matrix (A) [matrix of math nodes,%
             nodes = {circle},%
             left delimiter  = {[},%
             right delimiter = {]}] at (0,0)
{%
  a_{11} & a_{12} & \ldots & a_{1n}  \\
  \node[draw] {a_{21}};%
         & \node[draw] {a_{22}};%
                  & \ldots%
                           & \node[draw] {a_{2n}}; \\
  \vdots & \vdots & \ddots & \vdots  \\
  a_{m1} & a_{m2} & \ldots & a_{mn}  \\
};
v\matrix (v) [xshift=1cm,matrix of math nodes,nodes = {circle},% 
              left delimiter  = [,right delimiter ={]}] at (A.east)
{%
  \node[draw] {v_1};\\
  \node[draw] {v_2};\\
                       \vdots \\
  \node[draw] {v_n};\\
};
\node[circle,xshift=0.5cm] (E) at (v.east) {$=$}; 
w\matrix (w) [xshift=3.5cm,%
             matrix of math nodes,%
             nodes = {ellipse},%
             left delimiter  = [,%
             right delimiter ={]}] at (E.east)
{%
  a_{11}v_1+a_{12}v_2+\cdots+a_{1n}v_n\\\\
  \node[draw,blue] (X) {a_{21}v_1+a_{22}v_2+\cdots+a_{2n}v_n};\\
  \vdots \\
  a_{m1}v_1+a_{m2}v_2+\cdots+a_{mn}v_n\\
};
\draw[<->,red](A-2-1) to[in=180,out=90]
node[draw,sloped,midway,fill=white] (x) {$a_{21}v_{1}$} (v-1-1);
\draw[<->,red](A-2-2) to[in=180,out=-90]
node[draw,sloped,midway,fill=white] (y) {$a_{22}v_{2}$} (v-2-1);
\draw[<->,red](A-2-4) to[in=180,out=270]
node[draw,sloped,midway,fill=white] (z) {$a_{2n}v_{n}$} (v-4-1);
\draw[->,blue] (x) to (X);
\draw[->,blue] (y) to (X);
\draw[->,blue] (z) to (X);
\end{tikzpicture}


\end{remark}
\begin{proof}
Suppose $A=[a_{ij}]=[\vec{a}_1,\vec{a}_2,\ldots,\vec{a}_n]$ and  
$\vec{v} \in \mathbb{R}^n$. Then 
\[
w=A\vec{v}=[\vec{a}_1,\vec{a}_2,\ldots,\vec{a}_n] \begin{bmatrix}v_1\\ 
v_2 \\ 
\vdots \\ 
v_n\end{bmatrix}=v_1\vec{a}_1+v_2\vec{a}_2+\cdots+v_n\vec{a}_n
\]
Let $k$ be an integer with $1 \le k \le n$. Since 
$w=A\vec{v}=v_1\vec{a}_1+v_2\vec{a}_2+\cdots+v_n\vec{a}_n$ as vectors they are 
also equal at the $k^{\tiny\text{th}}$ coordinate. That is, 
$w(k)=v_1\vec{a}_1(k)+v_2\vec{a}_2(k)+\cdots+v_n\vec{a}_n(k)$. 
Note that the column vector 
$\vec{a}_j=\begin{bmatrix}a_{1j}\\ a_{2j}\\ \vdots \\ a_{nj}\end{bmatrix}$ for 
each $j$. That is, $\vec{a}_j(k)=a_{kj}$. Therefore, 
\[\vec{w}(k)=v_1a_{k1}+v_2a_{k2}+\cdots+v_na_{kn}=a_{k1}v_1+a_{k2}v_2+\cdots+a_{
kn}v_n.\]
\end{proof}

\subsubsection{Exercises}

\begin{exercise}For matrix 
$A=\begin{bmatrix*}[C]
1  & 2  \\
0  & -1  \\
-1 & 0  \\
\end{bmatrix*}$
Use the definition, as in \\Examples~\ref{ex:mv_mult_by_def}
and \ref{ex:mv_mult_by_def_gen}, to  multiply by the 
following vectors(If not possible: Why?):\\
\begin{inparaenum}[a.)]
\item $\begin{bmatrix*}[C]1 \\ -2 \\ -1 \end{bmatrix*}$\hspace{1cm}
\item $\begin{bmatrix*}[C]2 \\ -3 \end{bmatrix*}$\hspace{1cm}
\item $\begin{bmatrix*}[C]0 \\ 0 \\ 0 \end{bmatrix*}$\hspace{1cm}
\item $\begin{bmatrix*}[C]v_1 \\ v_2 \end{bmatrix*}$
\end{inparaenum}
\end{exercise}

\begin{exercise}For matrix 
$A=\begin{bmatrix*}[C]
1  & 2  & 3 & 4 & 5 \\
0  & -1 & 2 & 0 & 3 \\
-1 & 0  & 0 & -1& 1 \\
\end{bmatrix*}$
Use the \textbf{Row-Column Algorithm}, as in 
Proposition~\ref{ex:mv_mult_by_def}
to  multiply by the following vectors(If not possible: Why?):\\
\begin{inparaenum}[a.)]
\item $\begin{bmatrix*}[C]1 \\ -2 \\ -1 \\ 2 \\ 3\end{bmatrix*}$\hspace{1cm}
\item $\begin{bmatrix*}[C]1 \\ -2 \\ -1 \\ 2\end{bmatrix*}$\hspace{1cm}
\item $\begin{bmatrix*}[C]0 \\ 0 \\ 3 \\ 0 \\ 1\end{bmatrix*}$\hspace{1cm}
\item $\begin{bmatrix*}[C]v_1 \\ v_2 \\ v_3 \\ v_4 \\ v_5\end{bmatrix*}$
\end{inparaenum}
\end{exercise}

\begin{exercise} Consider
$A=\begin{bmatrix*}[C]
1 & 3  \\
0 & -1 \\
-3 & 2 \\ 
\end{bmatrix*}$\\
\begin{inparaenum}[a.)]
\item If $A\vec{v}$ is possible, for what value of $n$ is 
$\vec{v} \in \mathbb{R}^n$? (Why?)\\
%
\item If $A\vec{v}$ is possible, for what value of $m$ is 
$A\vec{v} \in \mathbb{R}^m$? (Why?)\\
%
\item Pick a generic vector $\vec{v} \in \mathbb{R}^n$ and write it as 
$\vec{v}=\begin{bmatrix}v_1 \\ v_2 \\ \vdots \\ v_n\end{bmatrix}$ where $n$ is from part a.) . What is the value of $A\vec{v}$?
\end{inparaenum}
\end{exercise}

\begin{exercise}
$A=\begin{bmatrix*}[C]
1 & 1  \\
0 & -3  \\
-1 & 1  
\end{bmatrix*}$\\
\begin{inparaenum}
\item If possible, find a vector $\vec{v}$ such that 
$A\vec{v}=\begin{bmatrix*}[C] 3 \\ -6 \\ 2\end{bmatrix*}$\\
%
\item If possible, find a vector $\vec{v}$ such that 
$A\vec{v}=\begin{bmatrix*}[C] 2 \\ -3 \\ 1\end{bmatrix*}$\\
\end{inparaenum}
\end{exercise}

\begin{exercise}
$A=\begin{bmatrix*}[C]
2 & 1  & -3\\
0 & -3 & 1 \\
0 & 0 & 1 
\end{bmatrix*}$\\
\begin{inparaenum}
\item If possible, find a vector $\vec{v}$ such that 
$A\vec{v}=\begin{bmatrix}1 \\ 2 \\ 3\end{bmatrix}$\\
%
\item Given any $\vec{b} \in \mathbb{R}^3$ can you always find a 
$\vec{v} \in \mathbb{R}^3$ such that $A\vec{v}=\vec{b}$? (Why?)
\end{inparaenum}
\end{exercise}

