\section{Introduction}
\index{$\mathbb{R}^n$, Introduction}
\subsection{$\mathbb{R}^n$}
We will assume that all readers are already familiar with a concept of real and complex numbers. In this text we denote the real numbers by $\mathbb{R}$ and the complex numbers by $\mathbb{C}$. 

\begin{definition}
We define $\mathbb{R}^n$ to be the set of real column vectors in $n$-coordinates. That is 
\[
\mathbb{R}^n=\left\{ \begin{pmatrix}
x_1\\ x_2 \\ \vdots \\ x_n
\end{pmatrix}
\mid
x_i \in \mathbb{R}\text{ for all }i=1,2, \ldots, n
\right\}
\]

We will refer to the members of $\mathbb{R}^n$ as \emph{vectors} and the elements of $\mathbb{R}$ as \emph{scalars}. When we want to refer to row vectors instead of column vectors we will use the notation $\mathbb{R}^n_\text{row}$.
\end{definition}

For convience we may refer to each $\vec{x} \in \mathbb{R}^n$ as a single bold letter where we will for allow each coordinate be labeled as the same subscriped letter. For example
\[
\vec{x}=\begin{pmatrix}x_1\\ x_2 \\ \vdots \\ x_n\end{pmatrix}
\vec{y}=\begin{pmatrix}y_1\\ y_2 \\ \vdots \\ y_n\end{pmatrix}
\vec{v}=\begin{pmatrix}v_1\\ v_2 \\ \vdots \\ v_n\end{pmatrix}
\]

\begin{remark}
Sometimes it is useful to describe a vector as a function on it's coordinates. For example if 
$\vec{v}=\left(\begin{array}{r}1 \\ 4 \\ 9
\end{array}\right)$ We may say $\vec{v}(2)=4$ that is the 2nd coordinate of $\vec{v}$ is $4$.
\end{remark}


\begin{definition}\index{vector equality}
Vectors $\vec{v}, \vec{w} \in \mathbb{R}^n$ are \emph{equal},  if they are equal coordinatewise. That is, $v_i=v(i)=w(i)=w_i$ for all $i=1,2,\ldots, n$. We denote this with  $\vec{v}=\vec{w}$.
\end{definition}


\begin{definition}\index{vector addition}
We define \emph{vector addition} on $\mathbb{R}^n$ coordinatewise. That is for $\vec{v},\vec{w} \in \mathbb{R}^n$ we define
\[
\vec{v}+\vec{w}=
\begin{pmatrix}v_1\\ v_2 \\ \vdots \\ v_n\end{pmatrix}+
\begin{pmatrix}w_1\\ w_2 \\ \vdots \\ w_n\end{pmatrix}=
\begin{pmatrix}v_1+w_1\\ v_2+w_2 \\ \vdots \\ v_n+w_n\end{pmatrix}
\]
\end{definition}

\begin{definition}]\index{scalar multiplcation}
We define \emph{scalar multiplcation} on $\mathbb{R}^n$ coordinatewise. That is for $\vec{v} \in \mathbb{R}^n$ and $r \in \mathbb{R}$ we define
\[
r\vec{v}=
r\begin{pmatrix}v_1\\ v_2 \\ \vdots \\ v_n\end{pmatrix}=
\begin{pmatrix}rv_1\\ rv_2 \\ \vdots \\ rv_n\end{pmatrix}
\]
\end{definition}

The order of operations is scalar multiplication first then vector addition.

\begin{example}
Consider  $\vec{x}=\begin{pmatrix}1\\ 0  \\ -3\end{pmatrix}$ and 
$\vec{y}=\begin{pmatrix}-4\\ 2 \\ 1\end{pmatrix}$ in $\mathbb{R}^3$.

\[
2\vec{x}+(-1)\vec{y}=
2\begin{pmatrix}1\\ 0  \\ -3\end{pmatrix}+
(-1)\begin{pmatrix}-4\\ 2 \\ 1\end{pmatrix}=
%\begin{pmatrix}2(1)\\ 2(0)  \\ 2(-3)\end{pmatrix}+
%\begin{pmatrix}(-1)(-4)\\ (-1)(2) \\ (-1)(1)\end{pmatrix}=
\begin{pmatrix}2\\ 0  \\ -6\end{pmatrix}+
\begin{pmatrix}4\\ -2 \\ -1\end{pmatrix}=
\begin{pmatrix}2+4\\ 0+(-2)  \\ -6+(-1)\end{pmatrix}=
\begin{pmatrix}6\\ -2  \\ -5\end{pmatrix}
\]
This is actually also an example of our next definition.
\end{example}
\begin{definition}\index{linear combination}
A \emph{linear combination} of vectors $\vec{a}_1, \vec{a}_2, \ldots, \vec{a}_n$ is 
\[
\begin{pmatrix}v_1 \\ v_2 \\ v_3\end{pmatrix}=x_1\vec{a}_1+x_2\vec{a}+\cdots+x_n\vec{a}_n
\]  
for some choice of scalar multiples $x_1,x_2, \ldots, x_n \in \mathbb{R}$
\end{definition}


\begin{example}
Note that any vector in $\mathbb{R}^3$ can be written as a linear combination
\[
\begin{pmatrix}v_1 \\ v_2 \\ v_3\end{pmatrix}=
v_1 \begin{pmatrix}1 \\ 0 \\ 0\end{pmatrix}+
v_2 \begin{pmatrix}0 \\ 1 \\ 0\end{pmatrix}+
v_3 \begin{pmatrix}0 \\ 0 \\ 1\end{pmatrix}
\]
\end{example}

\begin{definition}
\index{standard basis for $\mathbb{R}^n$}
The \emph{standard basis for} $\mathbb{R}^n$ is a collection of $n$-vectors $\vec{e}_1, \vec{e}_2, \ldots \vec{e}_n$ such that
$\vec{e}_j(k)=\begin{cases}
1 & \text{ if } k=j\\
0 & \text{ if } k\neq j
\end{cases}$ 
for each $j,k \in \{1, 2, \ldots, n\}$
\end{definition}

\begin{example}
In $\mathbb{R}^3 $ every vector is a linear combination of the standard basis $\vec{e}_1,\vec{e}_2,\vec{e}_3$
\[
\begin{pmatrix}v_1 \\ v_2 \\ v_3\end{pmatrix}=
v_1 \begin{pmatrix}1 \\ 0 \\ 0\end{pmatrix}+
v_2 \begin{pmatrix}0 \\ 1 \\ 0\end{pmatrix}+
v_3 \begin{pmatrix}0 \\ 0 \\ 1\end{pmatrix}
=v_1\vec{e}_1+v_3\vec{e}_3+v_3\vec{e}_3
\]
\end{example}

\begin{example}
In $\mathbb{R}^n$ every vector is a linear combination of the standard basis $\vec{e}_1, \vec{e}_2, \ldots \vec{e}_n$.
\begin{proof}
Let $\vec{v} \in \mathbb{R}^n$. 
Note $\vec{v}(k)=v_k=v_k\cdot 1=v_k \vec{e}_k(k)$ for all $k = 1,2,\ldots,n$. 
We can also add in $k-1$ zeros before and $n-k$ zeros after
$\vec{v}{k}=0+ \cdots + 0+v_k\vec{e}_k(k)+0+\cdots+0$ for all $k = 1,2,\ldots,n$.  
Since $\vec{e}_j(k)=0$ for all $j\neq k$ $\vec{v}(k)=v_1\vec{e}_1(k)+v_2\vec{e}_2(k)+\cdots+ v_k\vec{e}_k(k)+ \cdots + v_n\vec{e}_n(k)$  for all $k = 1,2,\ldots,n$.  
\end{proof}
\end{example}

\subsection{Real Matrices}
\index{Real Matrices}

\begin{definition}
A real matrix $A$ is a rectangular array of real numbers $a_{i,j} \in \mathbb{R}$ (usually the comma is omitted $a_{ij}$)  where the $i$ is the row and $j$ is the column number. That is:
\[
A=[a_{ij}]=[\vec{a}_1 \vec{a}_2 \ldots \vec{a}_n]=
\begin{bmatrix}
a_{11} & a_{12} & \cdots & a_{1n} \\
a_{21} & a_{22} & \cdots & a_{2n} \\
\vdots & \vdots & \ddots & \vdots \\
a_{m1} & a_{m2} & \cdots & a_{mn} 
\end{bmatrix}
\]
The \emph{dimensions} of the matrix are $m\times n$ where $m$ is the number of rows and $n$ is the number of columns. The vectors $\vec{a}_1, \vec{a}_2, \ldots, \vec{a}_n \in \mathbb{R}^n$ are the \emph{column vectors} of $A$. In particular, 
\[
\vec{a}_1=\begin{pmatrix}a_{11}\\a_{21}\\ \vdots \\ a_{m1}\end{pmatrix},
\vec{a}_2=\begin{pmatrix}a_{12}\\a_{22}\\ \vdots \\ a_{m2}\end{pmatrix}, \cdots,
\vec{a}_k=\begin{pmatrix}a_{1k}\\a_{2k}\\ \vdots \\ a_{mk}\end{pmatrix}, \cdots,
\vec{a}_n=\begin{pmatrix}a_{1n}\\a_{2n}\\ \vdots \\ a_{mn}\end{pmatrix}
\]
\end{definition}
\begin{example}
$A=\left[ \begin{array}{rrr}
2 & -3 & 0 \\
-5 & 0 & 1 \end{array}\right]$
is a $2 \times 3$ matrix $a_{12}=-3$ and  $\vec{a}_1=\left(\begin{array}{r} 2 \\ -5 \end{array}\right)$
\end{example}

