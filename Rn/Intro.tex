\section{Introduction}
\index{$\mathbb{R}^n$, Introduction}
\subsection{$\mathbb{R}^n$}
We will assume that all readers are already familiar with a concept of real and complex numbers. In this text we denote the real numbers by $\mathbb{R}$ and the complex numbers by $\mathbb{C}$. 

\begin{definition}
We define $\mathbb{R}^n$ to be the set of real column vectors in $n$-coordinates. That is 
\[
\mathbb{R}^n=\left\{ \begin{bmatrix}
x_1\\ x_2 \\ \vdots \\ x_n
\end{bmatrix}
\mid
x_i \in \mathbb{R}\text{ for all }i=1,2, \ldots, n
\right\}
\]

We will refer to the members of $\mathbb{R}^n$ as \emph{vectors} and the 
elements of $\mathbb{R}$ as \emph{scalars}. When we want to refer to row vectors 
instead of column vectors we will use the notation $\mathbb{R}^n_\text{row}$.
\end{definition}

For convenience we may refer to each $\vec{x} \in \mathbb{R}^n$ as a single 
bold letter where we will for allow each coordinate be labeled as the same 
subscriped letter. For example
\[
\vec{x}=\begin{bmatrix}x_1\\ x_2 \\ \vdots \\ x_n\end{bmatrix}
\vec{y}=\begin{bmatrix}y_1\\ y_2 \\ \vdots \\ y_n\end{bmatrix}
\vec{v}=\begin{bmatrix}v_1\\ v_2 \\ \vdots \\ v_n\end{bmatrix}
\]

\begin{remark}
Sometimes it is useful to describe a vector as a function on it's coordinates. 
For example if 
$\vec{v}=\begin{bmatrix}1 \\ 4 \\ 9 \end{bmatrix}$ We may say 
$\vec{v}(2)=4$ that is the 2nd coordinate of $\vec{v}$ is $4$.
\end{remark}


\begin{definition}\index{vector equality}
Vectors $\vec{v}, \vec{w} \in \mathbb{R}^n$ are \emph{equal},  if they are equal 
coordinatewise. That is, $v_i=v(i)=w(i)=w_i$ for all $i=1,2,\ldots, n$. We 
denote this with  $\vec{v}=\vec{w}$.
\end{definition}


\begin{definition}\index{vector addition}
We define \emph{vector addition} on $\mathbb{R}^n$ coordinatewise. That is for 
$\vec{v},\vec{w} \in \mathbb{R}^n$ we define
\[
\vec{v}+\vec{w}=
\begin{bmatrix}v_1\\ v_2 \\ \vdots \\ v_n\end{bmatrix}+
\begin{bmatrix}w_1\\ w_2 \\ \vdots \\ w_n\end{bmatrix}=
\begin{bmatrix}v_1+w_1\\ v_2+w_2 \\ \vdots \\ v_n+w_n\end{bmatrix}
\]
\end{definition}

\begin{definition}]\index{scalar multiplication}
We define \emph{scalar multiplication} on $\mathbb{R}^n$ coordinatewise. That 
is for $\vec{v} \in \mathbb{R}^n$ and $r \in \mathbb{R}$ we define
\[
r\vec{v}=
r\begin{bmatrix}v_1\\ v_2 \\ \vdots \\ v_n\end{bmatrix}=
\begin{bmatrix}rv_1\\ rv_2 \\ \vdots \\ rv_n\end{bmatrix}
\]
\end{definition}

The order of operations is scalar multiplication first then vector addition.

\begin{example}
Consider  $\vec{x}=\begin{bmatrix}1\\ 0  \\ -3\end{bmatrix}$ and 
$\vec{y}=\begin{bmatrix}-4\\ 2 \\ 1\end{bmatrix}$ in $\mathbb{R}^3$.

\[
2\vec{x}+(-1)\vec{y}=
2\begin{bmatrix}1\\ 0  \\ -3\end{bmatrix}+
(-1)\begin{bmatrix}-4\\ 2 \\ 1\end{bmatrix}=
%\begin{bmatrix}2(1)\\ 2(0)  \\ 2(-3)\end{bmatrix}+
%\begin{bmatrix}(-1)(-4)\\ (-1)(2) \\ (-1)(1)\end{bmatrix}=
\begin{bmatrix}2\\ 0  \\ -6\end{bmatrix}+
\begin{bmatrix}4\\ -2 \\ -1\end{bmatrix}=
\begin{bmatrix}2+4\\ 0+(-2)  \\ -6+(-1)\end{bmatrix}=
\begin{bmatrix}6\\ -2  \\ -5\end{bmatrix}
\]
This is actually also an example of our next definition.
\end{example}
\begin{definition}\index{linear combination}
A \emph{linear combination} of vectors $\vec{a}_1, \vec{a}_2, \ldots, \vec{a}_n$ 
is 
\[
\begin{bmatrix}v_1 \\ v_2 \\ 
v_3\end{bmatrix}=x_1\vec{a}_1+x_2\vec{a}+\cdots+x_n\vec{a}_n
\]  
for some choice of scalar multiples $x_1,x_2, \ldots, x_n \in \mathbb{R}$
\end{definition}


\begin{example}
Note that any vector in $\mathbb{R}^3$ can be written as a linear combination
\[
\begin{bmatrix}v_1 \\ v_2 \\ v_3\end{bmatrix}=
v_1 \begin{bmatrix}1 \\ 0 \\ 0\end{bmatrix}+
v_2 \begin{bmatrix}0 \\ 1 \\ 0\end{bmatrix}+
v_3 \begin{bmatrix}0 \\ 0 \\ 1\end{bmatrix}
\]
\end{example}

\begin{definition}
\index{standard basis for $\mathbb{R}^n$}
The \emph{standard basis for} $\mathbb{R}^n$ is a set of $n$-vectors 
$\{\vec{e}_1, \vec{e}_2, \ldots, \vec{e}_n\}$ such that
$\vec{e}_j(k)=\begin{cases}
1 & \text{ if } k=j\\
0 & \text{ if } k\neq j
\end{cases}$ 
for each $j,k \in \{1, 2, \ldots, n\}$
\end{definition}

\begin{example}
In $\mathbb{R}^3 $ every vector is a linear combination of the standard basis 
$\{\vec{e}_1,\vec{e}_2,\vec{e}_3\}$
\[
\begin{bmatrix}v_1 \\ v_2 \\ v_3\end{bmatrix}=
v_1 \begin{bmatrix}1 \\ 0 \\ 0\end{bmatrix}+
v_2 \begin{bmatrix}0 \\ 1 \\ 0\end{bmatrix}+
v_3 \begin{bmatrix}0 \\ 0 \\ 1\end{bmatrix}
=v_1\vec{e}_1+v_3\vec{e}_3+v_3\vec{e}_3
\]
\end{example}

\begin{proposition}
In $\mathbb{R}^n$ every vector is a linear combination of the standard basis 
$\{\vec{e}_1, \vec{e}_2, \ldots, \vec{e}_n\}$.
\end{proposition}

\begin{proof}
Let $\vec{v} \in \mathbb{R}^n$ and $1\leq k\leq n$. Then we observe that  
\[\vec{v}(k)=v_k=v_k\cdot 1=v_k \vec{e}_k(k).\]
We can rewrite this equation by adding $k-1$ zeros before and $n-k$ zeros 
after $\vec{v}(k)$. That is,  
\[\vec{v}(k)=0+ \cdots + 0+v_k\vec{e}_k(k)+0+\cdots+0.\] 
Since $\vec{e}_j(k)=0$ for all $j\neq k$ we have 
\[\vec{v}(k)=v_1\vec{e}_1(k)+v_2\vec{e}_2(k)+\cdots+ v_k\vec{e}_k(k)+ \cdots + 
v_n\vec{e}_n(k).\]
Since the above equation is true for any $1\leq k\leq n$ we have,  
\[\vec{v}=v_1\vec{e}_1+v_2\vec{e}_2+\cdots+ v_n\vec{e}_n.\]
\end{proof}


\subsection{Real Matrices}
\index{Real Matrices}

\begin{definition}
A real matrix $A$ is a rectangular array of real numbers $a_{i,j} \in 
\mathbb{R}$ (usually the comma is omitted $a_{ij}$)  where the $i$ is the row 
and $j$ is the column number. That is:
\[
A=[a_{ij}]=[\vec{a}_1 \vec{a}_2 \ldots \vec{a}_n]=
\begin{bmatrix}
a_{11} & a_{12} & \cdots & a_{1n} \\
a_{21} & a_{22} & \cdots & a_{2n} \\
\vdots & \vdots & \ddots & \vdots \\
a_{m1} & a_{m2} & \cdots & a_{mn} 
\end{bmatrix}
\]
The \emph{dimension} of the matrix are $m\times n$ where $m$ is the number of 
rows and $n$ is the number of columns. 
\end{definition}
The matrix $A$ can be also seen as a collection of vectors. Namely, 
$A=[a_{ij}]=[\vec{a}_1 \vec{a}_2 \ldots \vec{a}_n]$ where 
 $\vec{a}_1, \vec{a}_2,\ldots, \vec{a}_n \in \mathbb{R}^n$ are the \emph{column 
vectors} of $A$. In particular, 
\[
\vec{a}_1=\begin{bmatrix}a_{11}\\a_{21}\\ \vdots \\ a_{m1}\end{bmatrix},
\vec{a}_2=\begin{bmatrix}a_{12}\\a_{22}\\ \vdots \\ a_{m2}\end{bmatrix}, \cdots,
\vec{a}_k=\begin{bmatrix}a_{1k}\\a_{2k}\\ \vdots \\ a_{mk}\end{bmatrix}, \cdots,
\vec{a}_n=\begin{bmatrix}a_{1n}\\a_{2n}\\ \vdots \\ a_{mn}\end{bmatrix}
\]

\begin{example}
$A=\begin{bmatrix}
2 & -3 & 0 \\
-5 & 0 & 1 
\end{bmatrix}$
is a $2 \times 3$ matrix, $a_{12}=-3$, and  
$\vec{a}_1=\begin{bmatrix} 2 \\ -5 
\end{bmatrix}$
\end{example}

\subsection{Matrix-Vector Multiplication}

\begin{definition}\index{Matrix-Vector Multiplication}
Given an $m \times n$ matrix $A=[\vec{a}_1 \vec{a}_2 \ldots \vec{a}_n]$ and column vector $\vec{v} \in \mathbb{R}^n$. We define the \emph{Matrix-Vector} product to be the linear combination 
\[
A\vec{v}=[\vec{a}_1 \vec{a}_2 \ldots \vec{a}_n]\begin{pmatrix}v_1 \\ v_2 \\ \vdots \\ v_n \end{pmatrix}=v_1\vec{a}_1+v_2\vec{a}_2+\cdots+v_n\vec{a}_n
\]
\end{definition} 
\begin{remark}
It is important to note that the product is only defined if the number of columns in $A$ matches the number of coordinates in $\vec{v}$.
\end{remark}

\begin{example}
\begin{align*}
\left[\begin{array}{rrr}
2 & {-3} & 0 \\
{-5} & 0 & 1 \end{array}\right]\begin{pmatrix}1\\ 2 \\ 3 \end{pmatrix}
&=1\left(\begin{array}{r}2\\{-5}\end{array}\right)+2\left(\begin{array}{r}{-3}\\0\end{array}\right)+3\left(\begin{array}{r}0\\1\end{array}\right)\\
&=\left(\begin{array}{r}2\\{-5}\end{array}\right)+\left(\begin{array}{r}{-6}\\ 0\end{array}\right)+\left(\begin{array}{r}0\\3\end{array}\right)\\
&=\left(\begin{array}{r}{-4}\\{-2}\end{array}\right)
\end{align*}
\end{example}

\begin{example}
\begin{align*}
\left[\begin{array}{rrr}
2 & {-3} & 0 \\
{-5} & 0 & 1 \end{array}\right]\begin{pmatrix}v_1\\ v_2 \\ v_3 \end{pmatrix}
&=v_1\left(\begin{array}{r}2\\{-5}\end{array}\right)+v_2\left(\begin{array}{r}{-3}\\0\end{array}\right)+v_3\left(\begin{array}{r}0\\1\end{array}\right)\\
&=\left(\begin{array}{r}2v_1\\{-5v_1}\end{array}\right)+\left(\begin{array}{r}{-3v_2}\\ 0\end{array}\right)+\left(\begin{array}{r}0\\v_3\end{array}\right)\\
&=\left(\begin{array}{c}2v_1-3v_2\\-5v_1+v_3\end{array}\right)
\end{align*}
\end{example}

