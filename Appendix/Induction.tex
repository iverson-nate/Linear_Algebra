\section{The Principle of Mathematical Induction}
\index{Induction}

\lineartransformationequivalences*

\begin{proof} ($2 \implies 3$)
\begin{inparaenum}
\item[\textbf{Basis Step:}] For $k=1$ let $a_1 \in \mathbb{R}$ and
$\vec{v}_1 \in \mathbb{R}^n$. Let $\vec{w} \in \mathbb{R}^n$ be any other
vector. Then
\begin{align*}
T(a_1\vec{v}_1)&=T(a_1\vec{v}_1+\vec{0})\\
&=T(a_1\vec{v}_1+0\vec{w})\\
&=a_1T(\vec{v}_1)+0T(\vec{w})\\
&=a_1T(\vec{v}_1)+\vec{0}\\
&=a_1T(\vec{v}_1)
\end{align*}

Since the choice of $a_1$ and $\vec{v}_1$ were arbitrary this equality holds
for all $a_1 \in \mathbb{R}$ and $\vec{v}_1 \in \mathbb{R}^n$.\\[10pt]
%                                                                                                                                                                       
\item[\textbf{Induction Hypothesis: }] Suppose \\
$T(a_1\vec{v}_1+a_2\vec{v}_2+\cdots+a_k\vec{v}_k)=                                                                                                                      
a_1T(\vec{v}_1)+a_2T(\vec{v}_2)+\cdots+a_kT(\vec{v}_k)$ for all integers $k$
with $1 \le k \le N$. \\[10pt]
%                                                                                                                                                                       
\item[\textbf{Induction: }]Let $a_1, \ldots, a_N,a_{N+1} \in \mathbb{R}$ and
$\vec{v}_1,\ldots,\vec{v}_N,\vec{v}_{N+1} \in \mathbb{R}^n$. Define
$\vec{v}=a_1\vec{v}_1+\cdots+a_N\vec{v}_N$.

Then by the induction hypothesis $T(\vec{v})=a_1T(\vec{v}_1)+\cdots+a_NT(\vec{v}_N)$. Therefore,

\begin{align*}
T(a_1\vec{v}_1+\cdots+a_N\vec{v}_N+a_{N+1}\vec{v}_{N+1})
&=T(1\vec{v}+a_{N+1}\vec{v}_{N+1})\\
&=1T(\vec{v})+a_{N+1}T(\vec{v}_{N+1})\\
&=a_1T(\vec{v}_1)+\cdots+a_NT(\vec{v}_N)+a_{N+1}T(\vec{v}_{N+1})
\end{align*}
Which proves this case by mathematical induction.
\end{inparaenum}
\end{proof}
